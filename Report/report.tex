\documentclass{article}

\usepackage[margin=1.5in]{geometry}

\setlength{\parindent}{0em}
\setlength{\parskip}{1em}

\begin{document}
\author{s1236818}
\title{Software Engineering Large Practical \\ Report}
\date{December 2014}
\maketitle

\section{Design}
\subsection{General Application Design}

As specified in my proposal, my application is a web-based multilayer battling game. Users can create an account and have their scores recorded, and are then ranked based on the score. The two main parts of this are the user account system (with rankings), and the battle system itself. I therefore focused on these parts of the website. 

The user system is split into several connected parts, including login/logout pages, a registration form, user details and account detail pages, and of course the actual ranking page. Although it is not totally obvious from the current website, the public page used to view any user account and the page used to view your own account are different - this means that it would be straightforward to add editing functionality to the latter page.

The game itself is a very simplified version of what was described in the proposal. It matches any two players who are searching for a battle partner, and each player has a win or loss recorded in their profile.

There also exists a third section of the website - the "mechs list". This is a simple section which I used mainly to learn about Django. It consists mainly of just simple database lookups, and has no bearing on the rest of the website, but I have left it in anyway.

\subsection{Technologies}

As planned in my proposal, the application back-end is written in python using Django, which proved to be a good choice. As a relatively mature framework, I found the internet to be full of documentation, tutorials, and support even when combined with other technologies - making it easy to learn even without much experience. Additionally, due to it being quite heavyweight, it includes many useful feature that sped up the development process such as the included development server, test suite, and admin site. On the front-end I used JavaScript with jQuery, which I found to be very useful for simplifying actions which would normally be hard for someone new to the language to write. I also used jQuery to perform AJAX requests to my back-end, allowing for the game to update without refreshing the page.

\subsection{Code Structure}

The back-end code is split between Django apps, each representing a separate functionality of the website. There are 4 of these: "battles", "home", "mechs" and "users". These refer to the 3 functionalities described above, in addition to a separate app for simply displaying the home page. Each of these apps have their own models, views, and templates, as is standard in Django. By splitting the website into these apps I was able to share functionality in relevant places to avoid repeating code, but could also focus on each individual part separately.

The front-end consists of a single base HTML template, which is extended by the other specific templates, and a single CSS file. This keeps the look and feel of the site consistent on every page. I also kept all of my JavaScript simple enough to stay in a single file, as only the battle functionality uses it.

%The expected minimum features are:
% A working web site.
% Implementation of whatever the competition you are ranking is.
% User accounts with associated rankings, using an appropriate ranking scheme.
%
%I would expect a reasonable report to include, but not be limited to, the following:
% A justification for design choices, ranging from the language(s) you used to the architecture
%of your application
% A description of your design
% A description of your testing strategy
% Any deficiences in the functionality
% Any deficiences in the implementation/design
% Make sure that I know of any non-obvious functionality
%{ Try not to have user-functionality that is non-obvious. But you may have, for exam-
%ple, a good backup or security strategy.
% Things you would do differently
% Things you are especially proud of
%
%Assessment is based on the following criteria:
% Use of source code control
% Documentation, including source comments
% Testing
% Maintainable code
% Your report
% Early submission
%
%
%- design
%	- actual user-side structure (?)
%	  (what is included, what is not)
%		- non obvious functionality
%	- techs
%	- code structure
%	- (adaptability - css etc)
%- development?
%	- research
%	- spread of work
%	- source control
%	- (interesting/good refactorings/choices)
%- testing (^?)
%- difficulties (?)
%- deficiencies
%	- compare to proposal
%- imrpovements (do different)
%- improvements (more time)
%- pride
%
%
%
%Your report is there to convince me:
%You have followed sound software development practices
%You have done an adequate amount of work
%You have made well reasoned decisions
%Any deficiences are well justified or
%You have now learnt what you should have done instead


\end{document}
