\documentclass{article}

\usepackage[margin=1.5in]{geometry}

\setlength{\parindent}{0em}
\setlength{\parskip}{1em}

\begin{document}
\author{s1236818}
\title{Software Engineering Large Practical \\ Report}
\date{December 2014}
\maketitle

\section{Design}
\subsection{General Application Design}

As specified in my proposal, my application is a web-based multilayer battling game. Users can create an account and have their scores recorded, and are then ranked based on the score. The two main parts of this are the user account system (with rankings), and the battle system itself. I therefore focused on these parts of the website. 

The user system is split into several connected parts, including login/logout pages, a registration form, user details and account detail pages, and of course the actual ranking page. Although it is not totally obvious from the current website, the public page used to view any user account and the page used to view your own account are different - this means that it would be straightforward to add editing functionality to the latter page.

The game itself is a very simplified version of what was described in the proposal. It matches any two players who are searching for a battle partner, and each player has a win or loss recorded in their profile.

There also exists a third section of the website - the "mechs list". This is a simple section which I used mainly to learn about Django. It consists mainly of just simple database lookups, and has no bearing on the rest of the website, but I have left it in anyway.

\subsection{Technologies}

As planned in my proposal, the application back-end is written in python using Django, which proved to be a good choice. As a relatively mature framework, I found the internet to be full of documentation, tutorials, and support even when combined with other technologies - making it easy to learn even without much experience. Additionally, due to it being quite heavyweight, it includes many useful feature that sped up the development process such as the included development server, test suite, and admin site. It also provides a simple user authentication system, allowing me to use this already-written code and not have to worry about the security of it. 

For my web server I simply used Django's built in development server, as this project is not intended for production. This made it much easier to set up and use. For my database I used SQLite, which is included with Django. While this is not as powerful or secure as a less lightweight database engine, it was easy to configure and required no extra installations, making it perfect for this development-only practical. I also used South for my database migrations, which allowed my to easily change the model structure without having to write my own migration queries, which could easily go wrong.

On the front-end I used JavaScript with jQuery, which I found to be very useful for simplifying actions which would normally be hard for someone new to the language to write. I also used jQuery to perform AJAX requests to my back-end, allowing for the game to update without refreshing the page.

\subsection{Code Structure}

The back-end code is split between Django apps, each representing a separate functionality of the website. There are 4 of these: "battles", "home", "mechs" and "users". These refer to the 3 functionalities described above, in addition to a separate app for simply displaying the home page. Each of these apps have their own models, views, and templates, as is standard in Django. By splitting the website into these apps I was able to share functionality in relevant places to avoid repeating code, but could also focus on each individual part separately.

The front-end consists of a single base HTML template, which is extended by the other specific templates, and a single CSS file. This keeps the look and feel of the site consistent on every page. I also kept all of my own JavaScript simple enough to stay in a single file, as only the battle functionality uses it. In addition to my JavaScript, I included a copy of the non-minimised version of jQuery's source. I used the non-minimised version as it can result in more sensible and helpful error messages, and efficiency is not a concern during development. I included it locally as the online hosted version which I know of are the minimised version.

\subsection{Adaptability}

Throughout the development of my code, I was careful to make sure that it would be easy to adapt and change. Aside from general readability, such as writing comments so that others could understand the code, this involved separation of concerns, keeping small functions that could easily be changed to modify the behaviour. For example, the users are ranked by a "score" field on their database entry, which is calculated whenever wins or losses are added. However the calculation of this score is kept separate and so could easily be changed, perhaps to adjust the weighting of wins vs losses. Even if a new developer did not want to use this score field, the ranking view is shown using a \texttt{rankedUsers()} function, which returns the users ordered by score. However this could be altered in any way to allow the ranking to display something totally different. 

Another example is in the battle app, when players have chosen their moves. When the server calculates the damage dealt to each player, it refers to separate function to do this, which could contain any formula desired if the battle system were to be expanded upon (for example if players had individual attack/defence stats). Additionally, this currently refers to an array of move matchups to get the damages (as a formula would overcomplicate it), which could be changed if a small change was to be made - for example an extra move added.
Finally, there are many features that I thought of for my proposal which would have been fun to implement, such as user levels and custom mechs and moves, however these were all stretch goals, as I was unsure how difficult creating the basic website would be. In the end, making the minimal version of what I described in my proposal took the whole time and I was not able to look at other features, however given more time I would very much like to make the game deeper and more fun to play.
On of the most useful places in which my application is adaptable is the front-end visuals. Due to the fact that the Django backend simply sends some variables to the template which can be inserted, it would be straightforward for another developer with a absic understanding of the Django template language to improve the HTML design. The CSS is also in a separate file, and so this could also be modified by another developer easily.

The adaptability of the application in these ways is a great property if other developers need to change parts of it without understanding the whole system, or if there is a need or desire to add more features later.By doing this I have followed a basic good software engineering practice, as required by this practical.

\section{Development}

Being almost entirely new to web development, I had to spend more time than the average person might researching technologies and how to use them. Due to this and the fact that it is a relatively small assignment (compared to real software engineering projects) with limited time available, I had to prioritise much of my development. I therefore focused on the main requirements, and left some areas lacking in polish. These areas consist mainly of small changes and additions, but together they would take up a significant amount of time, which I instead used to get the game and rankings working. By effectively using my time in this way, I was able to create a working website fitting the minimum criteria even with little experience.

Throughout the development I used source control, committing code whenever a significant change was made, and keeping logically separate changes in separate commits. This would allow me to undo an single change without affecting the rest of the program. I used a .gitignore file to keep binary files, virtualenv files, and the database out of the git repository. Additionally I used a .gitattributes file to keep the line endings consistent throughout the project, which was useful when working on both Linux and Windows machines. Overall I used good source control practices, another general good software engineering practice.

\section{Testing}

While developing my website, I created a large number of tests for each feature as I developed it. I created these using Django's automatic test suite, which simplified the process largely. I tested every view (on good and bad inputs), view subfunction, and model method. This not only allowed me to be more confident that the components were working, but also by keeping these tests I could immediately check that I had not broken any features while developing new ones by running them periodically. These checks would have taken a long time if I had chosen to write all of my tests at the end of the project.

I did not use test-driven development, instead choosing to write tests after writing new features. While this may not be the best way to develop, I found it to be preferable due to the fact that I was relatively unfamiliar with both the framework and the test suite. Writing tests for new features that I did not yet fully understand the implementation of would be difficult, and would ultimately require me to "write" the new features in my head before writing any tests for them. Having had the experience of this practical however, I think that I could be capable now of developing Django in a test-driven way.

I used the coverage.py tool to assist me in writing tests by showing test coverage. Although this metric is not great (as you cannot tell what the tests are actually doing on each line), it helps by flagging areas which have been completely untouched by tests, and may contain bugs. At the end of the practical, my test coverage was 98\% (as can be seen by running the test script).

Throughout my project I used extensive testing, something which would only become more important if the size of the project grew. This is an important software engineering practice, and I think is worthy of a high mark in this practical.

\section{Improvements}

If there was more time available to me, there are may ways in which I could improve the application. One of the largest issues is that the server does not log a player disconnecting from a battle. If a player leaves halfway through then the game is stuck, and if a player leaves before finding a partner the game will still match them with someone else at a later point (possibly even themselves). This would be solved by counting a player as "disconnected" if they did not contact the server for a certain period of time (during a game each player will  request battle status updates once every second). However it was not obvious to me how to implement this, and I did not have time to do it properly, so I decided to leave it as a possible improvement.

There are many ways in which the user system could be improved to make it seem more polished. For example, there is currently no functionality to edit the user profile (although this profile exists). There are also some pages where perhaps it would be better to check if the user is logged in - for example it does not make sense for a logged in user to register a new account. These features would be quick to add thanks to Django's built-in user system, but there are so many similar examples that I could have spent a huge amount of time of this aspect of the project, and not been able to make a working game.

While the flow of the application is somewhat straightforward when things go well, when errors occur or things are done wrongly, the output in this case is not very clear at the moment. In many cases a function simply returns "False" when an error occurs and "True" when all goes well. I had intended to replace this with custom exceptions which would describe the errors, however there was not enough time to add this. Similarly, when a user receives a 400 or 403 status code, there is currently no indication of what they did wrongly. Giving more feedback would a good addition, making the website more user-friendly.

Finally, there are many features that I thought of for my proposal which would have been fun to implement, such as user levels and custom mechs and moves, however these were all stretch goals, as I was unsure how difficult creating the basic website would be. In the end, making the minimal version of what I described in my proposal took the whole time and I was not able to look at other features, however given more time I would very much like to make the game deeper and more fun to play.

\section{Successes}

One of the things I am most proud of is the tests I wrote for this application. Before now I have never written any non-trivial tests, so it was an aspect of the practical I was unsure about. However I was able to write a large number of tests covering a high proportion of the code. These tests made writing new code much faster and easier, giving me confidence that what I was doing was working, and also (despite not using test-driven development) thinking of the tests helped me to write new features.

I am also happy with how smooth the gameplay turned out to be. For some time I was unsure how well the AJAX calls would work, and if I might have to simply reload the page at each action - in particular since I was unfamiliar with JavaScript, and completely new to jQuery. However this worked out well, and resulted in a fluid gameplay experience. I am also happy that I was able to get the online multiplayer working, even if there are still issues that would need to be addressed with this aspect. Having tested the game with others, despite the simplicity I found it to be quite fun to play.

% Any deficiences in the implementation/design
% Things you would do differently
% Things you are especially proud of













%The expected minimum features are:
% A working web site.
% Implementation of whatever the competition you are ranking is.
% User accounts with associated rankings, using an appropriate ranking scheme.
%
%I would expect a reasonable report to include, but not be limited to, the following:
% A justification for design choices, ranging from the language(s) you used to the architecture
%of your application
% A description of your design
% A description of your testing strategy
% Any deficiences in the functionality
% Any deficiences in the implementation/design
% Make sure that I know of any non-obvious functionality
%{ Try not to have user-functionality that is non-obvious. But you may have, for exam-
%ple, a good backup or security strategy.
% Things you would do differently
% Things you are especially proud of
%
%Assessment is based on the following criteria:
% Use of source code control
% Documentation, including source comments
% Testing
% Maintainable code
% Your report
% Early submission
%
%
%- design
%	- actual user-side structure (?)
%	  (what is included, what is not)
%		- non obvious functionality
%	- techs
%	- code structure
%	- (adaptability - css etc)
%- development?
%	- research
%	- spread of work
%	- source control
%	- (interesting/good refactorings/choices)
%- testing (^?)
%- difficulties (?)
%- deficiencies
%	- compare to proposal
%- imrpovements (do different)
%- improvements (more time)
%- pride
%
%
%
%Your report is there to convince me:
%You have followed sound software development practices
%You have done an adequate amount of work
%You have made well reasoned decisions
%Any deficiences are well justified or
%You have now learnt what you should have done instead


\end{document}
